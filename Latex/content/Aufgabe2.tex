\section*{Aufgabe 2}
\label{sec:Aufgabe2}
\subsection*{a)}
Die Abbildung \ref{2a} zeigt Zufallszahlen welche mit dem Box-Muller-Algorithmus gezogen worden sind.
Diese sind in guter Übereinstimmung mit der analytischen Gaußverteilung.

\begin{figure}
	\centering
	\includegraphics[width=0.8 \linewidth]{../Aufgabe2a.pdf}
	\caption{Zufallszahlen mit dem Box-Muller-Algorithmus und die analytische Form der Gaußverteilung mit $\mu = 0$ und $\sigma = 1$ (in rot).}
	\label{2a}
\end{figure}

\subsection*{b)}
Abbildung \ref{2b} zeigt die Verteilung von aufaddierten Zufallszahlen.
Die Anzahl an aufaddierten Zahlen folgt dabei
\begin{equation*}
	N = \sigma^2 \cdot 12 \,,
\end{equation*}
dabei ist $\mu = 0$.
Mit $\sigma = 1$ ergibt sich somit die Summe von 12 aufaddierten Zufallszahlen.
Diese Methode wird je nach verlangter Standartabweichung immer aufwendiger und ist somit ineffizienter als andere Methoden.
Außerdem ist bei einem direkten Vergleich mit dem Box-Muller-Algorithmus in Abb. \ref{2a} zu sehen, dass diese Methode nicht so präzise ist wie seine Konkurrenz, da diese sich mit hoher Summandenzahl erst einer Gaußverteilung annähert.
Allerdings kann so zunächst die grundlegende Aussage des zentralen Grenzwertsatzes bestätigt werden.

\begin{figure}
	\centering
	\includegraphics[width=0.8 \linewidth]{../Aufgabe2b.pdf}
	\caption{12 Aufaddierte Zufallszahlen und die analytische Form der Gaußverteilung mit $\mu = 0$ und $\sigma = 1$ (in rot).
	Die Zufallszahlen ergeben aufgrund des zentralen Grenzwertsatzes annähernd eine Gaußverteilung.}
	\label{2b}
\end{figure}

\FloatBarrier
\subsection*{c)}
Die Abbildung \ref{2c} zeigt die Verteilung der Zufallszahlen erzeugt mit dem Neumannschen Rückweisungsverfahren.
Dabei wurden die Zufallszahlen gleichverteilt aus einer Fläche gezogen mit\\ 
$A = {(x,y)|x \in \mathbb{R} \, \text{und} \, 0 < y < \alpha \exp{(-|x|)}}$.
Die verschiedenen Plots zeigen dies für verschiedene Werte von $\alpha$.
Zu sehen ist hierbei, dass für kleine $\alpha$ die Fläche zu klein wird und der Gauß nicht vollständig abgesampled werden kann.
Somit ist klar die Struktur der Exponentialfunktion erkennbar.
Für $\alpha$ größer $1$ funktioniert das Verfahren zufriedenstellend.
Insgesamt sollte das $\alpha$ nicht zu hoch gewählt werden.
Zwar wird so die Fläche größer jedoch sorgt eben jede auch für eine schlechtere Effizienz des Neumannschen Rückweisungsverfahren.

\newpage
\begin{figure}
	\centering
	\includegraphics[width=1.5 \linewidth,angle=90]{../Aufgabe2c2.pdf}
	\caption{Neumannsches Rückweisungsverfahren mit angepasster Fläche, für die Zufallszahlen. Die Skala für $N$ ist normiert und ist ebenso die Achse für die Theoriekurve.}
	\label{2c}
\end{figure}
\FloatBarrier

\subsection*{d)}
Schritt 0 (Normieren):
\begin{equation}
	p(x) = \frac{1}{\pi} \frac{1}{1+x^2}
	\label{p}
\end{equation}
ist bereits normiert, da
\begin{align*}
	&\int_{-\infty}^{\infty} \frac{1}{\pi} \frac{1}{1+x^2} dx \\
	&= \frac{1}{\pi} \left. \mathrm{arctan}(x) \right \vert_{-\infty}^{\infty}\\
	&= \frac{1}{\pi} \cdot \left(\frac{\pi}{2} + \frac{\pi}{2} \right) = 1
\end{align*}
Methode 1 (nach Skript):

Schritt 1 (Invertieren):
\begin{align*}
	y &= \frac{1}{\pi} \frac{1}{1+x^2} \\
	y \pi &= \frac{1}{1+x^2} \\
	\frac{1}{\pi y} &= 1+x^2 \\
	x(y) &= \pm \sqrt{\frac{1}{\pi y} - 1} = p^{-1}(y)
\end{align*}

Schritt 2 (Ableiten):
\begin{align*}
	p^{-1}(y)^{'} &= \frac{1}{2} \cdot \frac{1}{\pm \sqrt{\frac{1}{\pi y} - 1}} \cdot (-1) \cdot \frac{1}{\pi y^2}\\
	p^{-1}(y)^{'} &= \mp \frac{1}{\sqrt{(4\pi y^3-4\pi y^4)}}
\end{align*}

Schritt 3 (Einsetzen/Ausrechnen):
Da die vorausgehende Wahrscheinlichkeitsdichte $f(x) = 1$ ist ergibt sich für die gesuchte Verteilung:
\begin{equation*}
	\tilde{f}(y) = |p^{-1}(y)^{'}| \quad \text{mit} \quad y \in p([x_1,x_2]),
\end{equation*}
da $x$ in $[0,1[$ liegt gilt für $y \in ]0, \frac{1}{\pi}[$.
Dabei ergibt sich die Verteilung \ref{2d1}, welche allerdings fehlende Werte in der Mitte hat.
Es ist zu erkennen, dass die funktionelle Form korrekt erscheint für die Seitenbänder.
Da diese Lücke sehr merkwürdig und nicht wegzubekommen war, hier nun eine Methode (aus SMD), die uns geläufiger ist und auch funktioniert.
~\\
~\\
Methode 2 (nach SMD Vorlesung):\\
Hierbei sei $z$ die gleichverteilte Zahl aus $[0,1[$.
Außerdem ist \eqref{p} bereits normiert.
\\
Schritt 1 (Integrieren und Gleichsetzten):
\begin{align*}
	z &= \frac{1}{\pi} \int_{-\infty}^{x} \frac{1}{1+x^{'^2}} dx^{'}\\
	z &= \frac{1}{\pi} \left. \mathrm{arctan}(x^{'}) \right \vert_{-\infty}^{x}\\
	z &= \frac{\mathrm{arctan}(x)}{\pi} + \frac{1}{2} \, .
\end{align*}
Schritt 2 (Umformen):
\begin{align*}
	z &= \frac{\mathrm{arctan}(x)}{\pi} + \frac{1}{2}\\
	\left( z- \frac{1}{2} \right) \pi &= \mathrm{arctan}(x)\\
	x(z) &= \mathrm{tan}\left(\left( z- \frac{1}{2} \right)\pi\right) \, .
\end{align*}
Damit ergibt sich die Verteilung aus Abb. \ref{2d2}, welche deutlich besser aussieht und zu der gesuchten Verteilung passt.
\begin{figure}
	\centering
	\includegraphics[width=0.8 \linewidth]{../Aufgabe2d.pdf}
	\caption{Mit der Transformationsmethode aus dem Kierfeld Skript erzeugte Verteilung.}
	\label{2d1}
\end{figure}

\begin{figure}
	\centering
	\includegraphics[width=0.8 \linewidth]{../Aufgabe2dSMD.pdf}
	\caption{Mit der Transformationsmethode aus SMD erzeugte Verteilung.}
	\label{2d2}
\end{figure}