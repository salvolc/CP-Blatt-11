\section*{Aufgabe 1}
\label{sec:Aufgabe1}
\subsection*{a) + b)}

Wie zu erkennen ist, sind die Zufallszahlen für den linear kongruenten Zufallszahlengenerators (LKZG) mit den ersten Parametersatz nicht gut nutzbar.
Dies ist insbesonders an der "glatten" Verteilung in Abb. \ref{i} zu erkennen. 
Diese sollte im Idealfall leichte Fluktuationen haben, jedoch ist die Periodenlänge des LKZG so klein, dass die Zahlenfolge sich häufig wiederholt.
Außerdem ist eine Häufung der Zufallszahlen in dem ersten und letzten Bin zu erkennen welche ebenfalls unerwünscht sind.\\
Abb. \ref{ii} zeigt die Zufallszahlen des LKZG mit dem zweiten Parametersatz. 
Diese Verteilung zeigt ein besseres Verhalten der Zufallszahlen.
Die Verteilung ist nicht mehr "glatt" sondern fluktuiert leicht zwischen den Zufallszahlen hin und her.
Bei genauerem Betrachten fallen jedoch Unstimmigkeiten auf.
Mit der Annahme von Poisson Verteilungen in den Bins kann die statistische Unsicherheit betrachtet werden (der untere Teil der Plots).
Auffällig ist hierbei, dass fast alle Werte in einer Standartabweichung zum gemeinsamen Mittelwert liegen.\\
Die Zufallszahlen des Mersenne Twisters hingegen, welche in Abb. \ref{M} zu sehen ist, verteilen sich leicht stärker gestreut um diesen.
Bei der Verteilung selbst ergeben sich keine deutlichen Unterschiede, außer die etwas stärkere Fluktuation der Werte.


\begin{figure}
	\centering
	\includegraphics[width= 0.9\linewidth]{../Aufg1i.pdf}
	\caption{Zufallszahlen generiert mit dem ersten linear kongruenten Generators unter der Verwendung des ersten Parametersatzes.}
	\label{i}
\end{figure}

\begin{figure}
	\centering
	\includegraphics[width= 0.9\linewidth]{../Aufg1ii.pdf}
	\caption{Zufallszahlen generiert mit dem ersten linear kongruenten Generators unter der Verwendung des zweiten Parametersatzes.}
	\label{ii}
\end{figure}

\begin{figure}
	\centering
	\includegraphics[width= 0.9\linewidth]{../Aufg1M.pdf}
	\caption{Zufallszahlen generiert mit dem Mersenne Twister 19937.}
	\label{M}
\end{figure}
\FloatBarrier
\subsection*{c)}
Die Abbildungen \ref{c1}-\ref{c3} zeigen die Funktion
\begin{equation*}
	p(x) = n \, \sin^4{(\pi \, x)}
\end{equation*}
sowohl analytisch (in rot) als auch mit Hilfe der Zufallszahlengeneratoren aus a) und dem Neumannschen Rückweisungsverfahren generiert (als Histogramm).
Wird der LKZG mit dem ersten Parametersatz genutzt, so ergibt sich mit Neumann keine sinnvolle Verteilung für die gesuchte Funktion.
Dies liegt insbesondere an der Periodenlänge, da so dieselben wenigen diskreten Zufallszahlen wieder und wieder gezogen werden, was nicht ausreicht, um durch die Funktion "durchzusamplen".\\
Wird der zweite Parametersatz oder der Mersenne Twister genutzt, so ergibt sich eine recht ansehnliche Form der Verteilung, die gut mit der analytischen übereinstimmt.
Eine Aussage welche dieser beiden besser oder schlechter ist lässt sich ohne weiteres nicht zeigen.
(Mit $10^6$ Zufallszahlen und extrem großen Binning ergeben sich leichte Tendenzen zum Mersenne Twister als besseren Zufallszahlengenerator s.h. Screenshot bei Zusatzplots)

\begin{figure}
	\centering
	\includegraphics[width= 0.8\linewidth]{../Aufg1c1.pdf}
	\caption{Neumann Verfahren für die gesuchte Funktion mit Zufallszahlen generiert mit dem ersten linear kongruenten Generators unter der Verwendung des ersten Parametersatzes.}
	\label{c1}
\end{figure}

\begin{figure}
	\centering
	\includegraphics[width= 0.8\linewidth]{../Aufg1c2.pdf}
	\caption{Neumann Verfahren für die gesuchte Funktion mit Zufallszahlen generiert mit dem ersten linear kongruenten Generators unter der Verwendung des zweiten Parametersatzes.}
	\label{c2}
\end{figure}

\begin{figure}
	\centering
	\includegraphics[width= 0.8\linewidth]{../Aufg1c3.pdf}
	\caption{Neumann Verfahren für die gesuchte Funktion mit Zufallszahlen generiert mit dem Mersenne Twister 19937.}
	\label{c3}
\end{figure}

\FloatBarrier
\subsection*{d)}
Der 2D Spektraltest wird durchgeführt indem nacheinanderfolgende Glieder der Zufallszahlen gegeneinander aufgetragen werden.
Der Spektraltest ist dabei ein Test für die Zufälligkeit nacheinanderfolgender Zahlen.
Ein guter Pseudo Zufallszahlengenerator sollte keine Struktur haben.
Die kann für den zweiten LKZG und den Mersenne Twister in Abb. \ref{d2} und Abb. \ref{d3} gesehen werden.
Für einen schlechten Zufallszahlengenerator können Strukturen (im 2D Fall Geraden) gefunden werden.
Dies ist bei dem ersten LKZG der Fall, wir in Abb. \ref{d1} zu sehen ist.
\begin{figure}
	\centering
	\includegraphics[width= 0.8\linewidth]{../Aufg1di.png}
	\caption{2D Spektraltest für den ersten linear kongruenten Generators unter der Verwendung des ersten Parametersatzes. Geplottet werden jeweils nacheinander folgende Zahlen als Paare.}
	\label{d1}
\end{figure}

\begin{figure}
	\centering
	\includegraphics[width= 0.8\linewidth]{../Aufg1dii.png}
	\caption{2D Spektraltest für den ersten linear kongruenten Generators unter der Verwendung des zweiten Parametersatzes. Geplottet werden jeweils nacheinander folgende Zahlen als Paare.}
	\label{d2}
\end{figure}

\begin{figure}
	\centering
	\includegraphics[width= 0.8\linewidth]{../Aufg1dM.png}
	\caption{2D Spektraltest für den Mersenne Twister 19937. Geplottet werden jeweils nacheinander folgende Zahlen als Paare.}
	\label{d3}
\end{figure}
\FloatBarrier
\subsection*{d)}
Der 30 Spektraltest wird durchgeführt indem drei nacheinanderfolgende Glieder der Zufallszahlen gegeneinander aufgetragen werden.
Dieser Test ist analog zum 2D Test.
Für den Mersenne Twister kann der Spektraltest in Abb. \ref{e3} gesehen werden.
Bei diesen zeichnet sich keine Struktur ab.
Bei dem Spektraltest für ersten LKZG in Abb. \ref{e1} und dem zweiten LKZG in Abb. \ref{e2} zeichnen sich jedoch klar Ebenen ab, welches für einen schlechteren Pseudo Zufallszahlengenerator spricht.
Bei diesem Test zeichnet sich zum ersten Mal der Mersenne Twister deutlich von dem LKZG mit "gut" gewählten Parametern ab.
Würde die Dimension des Test erhöht, so würden sich ebenfalls für den Mersenne Twister Hyperbenen abzeichnen, für diesen Test hier zeichnet sich dieser als "bester" Zufallszahlengenerator aus.


\begin{figure}
	\centering
	\includegraphics[width= 0.8\linewidth]{../Aufg1ei.png}
	\caption{3D Spektraltest für den ersten linear kongruenten Generators unter der Verwendung des ersten Parametersatzes. Geplottet werden jeweils nacheinander folgende Zahlen als Tripel.}
	\label{e1}
\end{figure}

\begin{figure}
	\centering
	\includegraphics[width= 0.8\linewidth]{../Aufg1eii.png}
	\caption{3D Spektraltest für den ersten linear kongruenten Generators unter der Verwendung des zweiten Parametersatzes. Geplottet werden jeweils nacheinander folgende Zahlen als Tripel.}
	\label{e2}
\end{figure}

\begin{figure}
	\centering
	\includegraphics[width= 0.8\linewidth]{../Aufg1eM.png}
	\caption{3D Spektraltest für den Mersenne Twister 19937. Geplottet werden jeweils nacheinander folgende Zahlen als Tripel.}
	\label{e3}
\end{figure}